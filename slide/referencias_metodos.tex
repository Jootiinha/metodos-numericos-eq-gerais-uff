% Seção adicional: Referências e Métodos Relacionados
% Baseado em trabalhos típicos sobre Helmholtz

\section{Contexto e Métodos Relacionados}

\begin{frame}{Equação de Helmholtz: Contexto Histórico}
\begin{block}{Origem}
A equação de Helmholtz surge da separação de variáveis na equação da onda:
\begin{equation*}
\frac{\partial^2 u}{\partial t^2} = c^2 \Delta u \quad \Rightarrow \quad \Delta u + k^2 u = 0
\end{equation*}
onde $k = \omega/c$ é o número de onda.
\end{block}

\vspace{0.2cm}

\begin{block}{Importância}
\begin{itemize}
\item Modela fenômenos de propagação de ondas em regime estacionário
\item Base para problemas de espalhamento e radiação
\item Desafio numérico fundamental em computação científica
\end{itemize}
\end{block}
\end{frame}

\begin{frame}{Métodos Numéricos para Helmholtz}
\begin{columns}
\begin{column}{0.5\textwidth}
\begin{block}{Métodos Diretos}
\begin{itemize}
\item Diferenças Finitas (este trabalho)
\item Elementos Finitos
\item Volumes Finitos
\end{itemize}
\end{block}

\vspace{0.2cm}

\begin{block}{Métodos Espectrais}
\begin{itemize}
\item Métodos de Galerkin
\item Pseudospectral
\item Alta precisão, mas menos flexível
\end{itemize}
\end{block}
\end{column}

\begin{column}{0.5\textwidth}
\begin{block}{Vantagens DF}
\begin{itemize}
\item Simplicidade de implementação
\item Estrutura de matriz esparsa
\item Fácil paralelização
\item Boa para domínios retangulares
\end{itemize}
\end{block}

\vspace{0.2cm}

\begin{block}{Desvantagens DF}
\begin{itemize}
\item Requer malhas uniformes
\item Poluição numérica para $k$ grande
\item Menos flexível para geometrias complexas
\end{itemize}
\end{block}
\end{column}
\end{columns}
\end{frame}

\begin{frame}{Desafios Numéricos Específicos}
\begin{block}{Problema da Poluição Numérica}
Para $k$ grande, métodos de baixa ordem requerem:
\begin{equation*}
N_{\lambda} = \frac{2\pi}{kh} \gtrsim 10-20
\end{equation*}
Caso contrário, erro de fase acumula e solução degrada.
\end{block}

\vspace{0.2cm}

\begin{block}{Soluções na Literatura}
\begin{itemize}
\item \textbf{Métodos de alta ordem}: 4ª, 6ª ordem (reduz $N_{\lambda}$ necessário)
\item \textbf{Malhas adaptativas}: Refinamento local
\item \textbf{Pré-condicionadores especializados}: Otimizados para Helmholtz
\item \textbf{Métodos híbridos}: Combinam diferentes abordagens
\end{itemize}
\end{block}
\end{frame}

\begin{frame}{Comparação com Outros Trabalhos}
\begin{block}{Resultados Típicos na Literatura}
\begin{itemize}
\item \textbf{Ordem 2}: Taxa de convergência $p \approx 2.0$ para $k$ pequeno
\item \textbf{Poluição}: Taxa degrada para $p < 1$ quando $kh$ grande
\item \textbf{Requisito de malha}: $N_{\lambda} \geq 10$ mínimo, $N_{\lambda} \geq 20$ recomendado
\item \textbf{Erros relativos}: $10^{-6}$ a $10^{-8}$ alcançáveis para $k$ pequeno
\end{itemize}
\end{block}

\vspace{0.2cm}

\begin{block}{Nossos Resultados}
\begin{itemize}
\item Taxa experimental: $p \approx 2.0$ para $k=1$ (conforme esperado)
\item Taxa degrada para $p \approx 0.5-1.0$ para $k=100$ (poluição numérica)
\item Erros relativos: $10^{-7}$ a $10^{-8}$ para $k=1$ (excelente)
\item Comportamento consistente com literatura
\end{itemize}
\end{block}
\end{frame}
