% Seção: Discussão Detalhada

\section{Discussão}

\begin{frame}{Interpretação dos Resultados}
\begin{block}{Regime de Baixa Frequência ($k=1$)}
\begin{itemize}
\item \textbf{Taxa de convergência}: $p \approx 2.0$ (conforme esperado)
\item \textbf{Erro relativo}: $10^{-7}$ a $10^{-8}$ (excelente precisão)
\item \textbf{Comportamento}: Método funciona perfeitamente, erro dominado por truncamento
\item \textbf{Conclusão}: Diferenças finitas de 2ª ordem são adequadas
\end{itemize}
\end{block}

\vspace{0.2cm}

\begin{block}{Regime de Alta Frequência ($k=100$)}
\begin{itemize}
\item \textbf{Taxa de convergência}: $p \approx 0.5-1.0$ (degradada)
\item \textbf{Erro relativo}: $O(1)$ a $O(10)$ (inaceitável)
\item \textbf{Causa}: Poluição numérica - poucos pontos por comprimento de onda
\item \textbf{Solução}: Requer $N_{\lambda} \geq 20-30$, ou seja, $N \gtrsim 400-600$
\end{itemize}
\end{block}
\end{frame}

\begin{frame}{Análise da Poluição Numérica}
\begin{block}{Definição}
\textbf{Poluição Numérica}: Mesmo com refinamento de malha, o erro cresce com o aumento de $k$, devido à má representação da fase da onda.
\end{block}

\vspace{0.2cm}

\begin{block}{Mecanismo}
\begin{enumerate}
\item \textbf{Poucos pontos por comprimento de onda}: $N_{\lambda} < 20$
\item \textbf{Erro de fase}: Velocidade de fase numérica $\neq$ exata
\item \textbf{Acúmulo}: Erro se propaga e amplifica
\item \textbf{Resultado}: Solução numérica "atrasa" espacialmente
\end{enumerate}
\end{block}

\vspace{0.2cm}

\begin{block}{Consequência}
É necessário um número crescente de pontos por comprimento de onda para manter a precisão quando $k$ aumenta.
\end{block}
\end{frame}

\begin{frame}{Comparação com Literatura}
\begin{block}{Resultados Consistentes}
\begin{itemize}
\item \textbf{Ordem de convergência}: $p \approx 2.0$ para $k$ pequeno (conforme teoria)
\item \textbf{Requisito de malha}: $N_{\lambda} \geq 10-20$ (alinhado com literatura)
\item \textbf{Poluição numérica}: Observada para $kh$ grande (esperado)
\item \textbf{Erros relativos}: $10^{-4}$ para $k=1$ com $N=60$ (trabalho original)
\item \textbf{Erros relativos}: $10^{-7}$ a $10^{-8}$ para $k=1$ com $N \geq 128$ (extensão)
\end{itemize}
\end{block}

\vspace{0.2cm}

\begin{block}{Contribuições do Trabalho}
\begin{itemize}
\item Validação numérica completa (6 grupos, múltiplas malhas)
\item Análise sistemática da taxa de convergência
\item Implementação eficiente com matrizes esparsas
\item Análise do efeito de poluição numérica
\item Avaliação do uso de IA como ferramenta auxiliar
\end{itemize}
\end{block}
\end{frame}

\begin{frame}{Limitações e Desafios}
\begin{block}{Limitações do Método}
\begin{itemize}
\item \textbf{Poluição numérica}: Inerente a métodos de baixa ordem
\item \textbf{Custo computacional}: Cresce com $N^2$ (sistemas grandes)
\item \textbf{Geometria}: Limitado a domínios retangulares simples
\item \textbf{Alta frequência}: Requer malhas muito refinadas
\end{itemize}
\end{block}

\vspace{0.2cm}

\begin{block}{Desafios Enfrentados}
\begin{itemize}
\item \textbf{Sistemas grandes}: Necessidade de métodos iterativos
\item \textbf{Condicionamento}: Piora com $k$ grande
\item \textbf{Singularidades}: Tratamento de casos onde ILU falha
\item \textbf{Performance}: Otimização através de cache e paralelização
\end{itemize}
\end{block}
\end{frame}
