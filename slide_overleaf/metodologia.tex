% Seção 2: Metodologia

\section{Metodologia}

\begin{frame}{Discretização e Sistema Linear}
\begin{block}{MDFC2}
Diferenças finitas centradas de 2ª ordem no domínio $\Omega = [0,1]^2$:
\begin{equation*}
\frac{u_{i+1,j} - 2u_{i,j} + u_{i-1,j}}{h^2} + \frac{u_{i,j+1} - 2u_{i,j} + u_{i,j-1}}{h^2} + k^2 u_{i,j} = 0
\end{equation*}
Resulta em sistema esparso: $A \mathbf{u} = \mathbf{b}$ com $A = L + k^2 I$ ($(N-1)^2 \times (N-1)^2$)
\end{block}
\end{frame}

\begin{frame}{Solução: Matrizes Esparsas}
\begin{block}{Desafio}
$N = 200 \Rightarrow 40.000$ incógnitas. Matriz densa: $\approx 12$ GB RAM (inviável).
\end{block}

\vspace{0.3cm}

\begin{block}{Implementação}
\begin{itemize}
\item \textbf{Formato LIL}: Montagem eficiente da matriz (apenas 5 diagonais não-nulas)
\item \textbf{Conversão CSC}: Para solução otimizada (\texttt{scipy.sparse})
\item \textbf{Memória}: Irrisória comparada à matriz densa
\end{itemize}
\end{block}
\end{frame}

\begin{frame}{Solvers Implementados}
\begin{block}{SPLU - Método Direto}
\begin{itemize}
\item Fatoração LU esparsa (\texttt{scipy.sparse.linalg.splu})
\item Eficiente para $N \leq 192$: $O(N^3)$ operações, $O(N^2)$ memória
\end{itemize}
\end{block}

\vspace{0.2cm}

\begin{block}{GMRES+ILU - Método Iterativo}
\begin{itemize}
\item GMRES com pré-condicionador ILU (\texttt{scipy.sparse.linalg.gmres} + \texttt{spilu})
\item Eficiente para $N \geq 256$: $O(N^2)$ por iteração
\end{itemize}
\end{block}
\end{frame}

\begin{frame}{Seleção Automática}
\begin{block}{Modo Auto}
\begin{itemize}
\item Escolha automática do solver baseada no tamanho do sistema
\item Fallback robusto: se ILU falha, usa SPLU ou GMRES sem pré-condicionador
\item Garante solução para todos os casos
\end{itemize}
\end{block}
\end{frame}
