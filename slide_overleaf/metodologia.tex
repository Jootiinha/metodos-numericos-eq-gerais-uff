% Seção 2: Metodologia

\section{Metodologia}

\begin{frame}{Discretização da Equação de Helmholtz}
\begin{block}{Aplicando MDFC2}
Aplicando o Método das Diferenças Finitas Centradas de Segunda Ordem no domínio $\Omega = [0,1]^2$:
\begin{equation*}
\frac{u_{i+1,j} - 2u_{i,j} + u_{i-1,j}}{h^2} + \frac{u_{i,j+1} - 2u_{i,j} + u_{i,j-1}}{h^2} + k^2 u_{i,j} = 0
\end{equation*}
\end{block}
\end{frame}

\begin{frame}{Sistema Linear Resultante}
\begin{block}{Forma Matricial}
Resultando em um sistema linear esparso:
\begin{equation*}
A \mathbf{u} = \mathbf{b}
\end{equation*}
\end{block}

\vspace{0.3cm}

\begin{block}{Componentes}
\begin{itemize}
\item $A$: Matriz esparsa $(N-1)^2 \times (N-1)^2$
\item $\mathbf{u}$: Vetor de incógnitas (valores nos pontos internos)
\item $\mathbf{b}$: Vetor com contribuições das condições de contorno Dirichlet
\end{itemize}
\end{block}
\end{frame}

\begin{frame}{Desafio Computacional}
\begin{block}{Problema}
Para capturar a física em $k = 100$, a malha deve ser fina ($N > 200$).
\begin{itemize}
\item $N = 200 \Rightarrow 40.000$ incógnitas
\item Matriz densa: $\approx 12$ GB de RAM (inviável/lento)
\end{itemize}
\end{block}
\end{frame}

\begin{frame}{Solução: Matrizes Esparsas}
\begin{block}{Vantagens}
\begin{itemize}
\item \textbf{Formato LIL/CSC}: Armazena apenas elementos não-zero
\item \textbf{Consumo de memória}: Irrisório comparado à matriz densa
\item \textbf{Implementação}: Utilizando \texttt{scipy.sparse}
\end{itemize}
\end{block}

\vspace{0.3cm}

\begin{block}{Montagem Eficiente}
\begin{itemize}
\item Montagem da matriz em formato LIL (List of Lists)
\item Conversão para CSC (Compressed Sparse Column) para solução
\end{itemize}
\end{block}
\end{frame}

\begin{frame}{Sistema Linear}
\begin{block}{Forma Matricial}
O problema discretizado resulta em um sistema linear esparso:
\begin{equation*}
A \mathbf{u} = \mathbf{b}
\end{equation*}
onde:
\begin{itemize}
\item $A = L + k^2 I$ (matriz esparsa $(N-1)^2 \times (N-1)^2$)
\item $L$ é a discretização do Laplaciano (estrutura de banda)
\item $\mathbf{b}$ contém as contribuições das condições de contorno Dirichlet
\item Estrutura esparsa: apenas 5 diagonais não-nulas
\end{itemize}
\end{block}

\vspace{0.2cm}

\begin{block}{Propriedades da Matriz}
\begin{itemize}
\item \textbf{Simétrica}: Para condições de contorno apropriadas
\item \textbf{Definida positiva}: Para $k$ real (garante solução única)
\item \textbf{Condicionamento}: Piora com $k$ grande (requer métodos iterativos)
\end{itemize}
\end{block}
\end{frame}

\begin{frame}{Solvers Implementados}
\begin{block}{SPLU - Método Direto}
\begin{itemize}
\item Fatoração LU esparsa (direto)
\item Eficiente para sistemas pequenos/médios ($N \leq 192$)
\item Custo: $O(N^3)$ operações, $O(N^2)$ memória
\end{itemize}
\end{block}

\vspace{0.2cm}

\begin{block}{GMRES+ILU - Método Iterativo}
\begin{itemize}
\item Método iterativo com pré-condicionador
\item Eficiente para sistemas grandes ($N \geq 256$)
\item Custo: $O(N^2)$ por iteração, convergência rápida
\end{itemize}
\end{block}
\end{frame}

\begin{frame}{Seleção Automática e Robustez}
\begin{block}{Modo Auto}
Escolhe automaticamente o solver baseado no tamanho do sistema.
\end{block}

\vspace{0.3cm}

\begin{block}{Tratamento de Singularidades}
\begin{itemize}
\item Fallback automático quando ILU falha
\item Uso de SPLU ou GMRES sem pré-condicionador como alternativa
\item Garante robustez para todos os casos
\end{itemize}
\end{block}
\end{frame}

\begin{frame}{Solução Exata para Validação}
\begin{block}{Superposição de Ondas Planas}
Para validação e cálculo de erro, utilizamos solução exata imposta como condição de Dirichlet na fronteira $\Gamma$:
\begin{equation*}
u_{\text{exata}}(x, y) = \sum_{i=1}^{3} \cos\left(k (x \cos \theta_i + y \sin \theta_i)\right)
\end{equation*}
com ângulos: $\Theta = \{0, \pi/8, \pi/4\}$
\end{block}
\end{frame}

\begin{frame}{Parâmetros de Teste}
\begin{block}{Configuração do Experimento}
\begin{itemize}
\item Malhas: $N \in \{64, 128, 192, 256\}$ (extendido para análise completa)
\item Números de onda: $k \in \{1, 20, 40, 100\}$
\item Validação inicial: $N = 60$ (conforme trabalho original)
\item Total: $6 \times 4 \times 4 = 96$ casos (extensão do trabalho)
\end{itemize}
\end{block}
\end{frame}
