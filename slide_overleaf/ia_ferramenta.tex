% Seção: Utilização de Inteligência Artificial

\section{Inteligência Artificial como Ferramenta Auxiliar}

\begin{frame}{Utilização da Inteligência Artificial}
\begin{block}{Ferramentas Utilizadas}
\begin{itemize}
\item \textbf{ChatGPT}: Suporte conceitual e sugestões metodológicas
\item \textbf{Gemini}: Análise comparativa e validação de abordagens
\item \textbf{NotebookLM}: Organização e síntese de informações
\end{itemize}
\end{block}

\vspace{0.2cm}

\begin{block}{Aplicações}
As IAs foram utilizadas como ferramentas de apoio:
\begin{itemize}
\item \textbf{Conceitual}: Esclarecimento de dúvidas teóricas
\item \textbf{Comparativa}: Validação de diferentes abordagens
\item \textbf{Metodológica}: Sugestões de implementação
\end{itemize}
\end{block}
\end{frame}

\begin{frame}{Comparação entre IAs}
\begin{block}{Critérios de Avaliação}
\begin{itemize}
\item \textbf{Sugestão de métodos}: Qualidade das recomendações numéricas
\item \textbf{Clareza teórica}: Explicações e fundamentação
\item \textbf{Custo computacional sugerido}: Análise de eficiência
\end{itemize}
\end{block}

\vspace{0.2cm}

\begin{block}{Resultados da Comparação}
\begin{table}[h]
\centering
\small
\begin{tabular}{lccc}
\toprule
IA & Métodos & Teoria & Custo \\
\midrule
ChatGPT & Boa & Excelente & Razoável \\
Gemini & Excelente & Boa & Boa \\
NotebookLM & Razoável & Boa & Razoável \\
\bottomrule
\end{tabular}
\caption{Comparação qualitativa entre IAs (exemplo)}
\end{table}
\end{block}

\vspace{0.2cm}

\begin{alertblock}{Nota}
A IA mostrou-se útil como suporte acadêmico, complementando o trabalho de pesquisa e implementação.
\end{alertblock}
\end{frame}
