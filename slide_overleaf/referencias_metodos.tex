% Seção adicional: Referências e Métodos Relacionados
% Baseado em trabalhos típicos sobre Helmholtz

\section{Contexto e Métodos Relacionados}

\begin{frame}{Contexto e Relevância}
\begin{block}{Importância na Computação Científica}
\begin{itemize}
\item Desafio numérico fundamental em problemas de propagação de ondas
\item Base para problemas de espalhamento, radiação e acústica
\item Requer métodos eficientes para $k$ grande (alta frequência)
\end{itemize}
\end{block}

\vspace{0.3cm}

\begin{block}{Abordagens na Literatura}
Diversos métodos numéricos têm sido desenvolvidos e estudados para resolver a equação de Helmholtz, cada um com suas vantagens e limitações.
\end{block}
\end{frame}

\begin{frame}{Métodos Numéricos para Helmholtz}
\begin{columns}
\begin{column}{0.5\textwidth}
\begin{block}{Métodos Diretos}
\begin{itemize}
\item Diferenças Finitas (este trabalho)
\item Elementos Finitos
\item Volumes Finitos
\end{itemize}
\end{block}

\vspace{0.2cm}

\begin{block}{Métodos Espectrais}
\begin{itemize}
\item Métodos de Galerkin
\item Pseudospectral
\item Alta precisão, mas menos flexível
\end{itemize}
\end{block}
\end{column}

\begin{column}{0.5\textwidth}
\begin{block}{Vantagens DF}
\begin{itemize}
\item Simplicidade de implementação
\item Estrutura de matriz esparsa
\item Fácil paralelização
\item Boa para domínios retangulares
\end{itemize}
\end{block}

\vspace{0.2cm}

\begin{block}{Desvantagens DF}
\begin{itemize}
\item Requer malhas uniformes
\item Poluição numérica para $k$ grande
\item Menos flexível para geometrias complexas
\end{itemize}
\end{block}
\end{column}
\end{columns}
\end{frame}

\begin{frame}{Desafios Numéricos Específicos}
\begin{block}{Problema da Poluição Numérica}
Para $k$ grande, métodos de baixa ordem requerem:
\begin{equation*}
N_{\lambda} = \frac{2\pi}{kh} \gtrsim 10-20
\end{equation*}
Caso contrário, erro de fase acumula e solução degrada (Babuška \& Sauter, 1997).
\end{block}

\vspace{0.2cm}

\begin{block}{Soluções na Literatura}
\begin{itemize}
\item \textbf{Métodos de alta ordem}: 4ª, 6ª ordem (reduz $N_{\lambda}$ necessário) (Ihlenburg \& Babuška, 1995)
\item \textbf{Malhas adaptativas}: Refinamento local
\item \textbf{Pré-condicionadores especializados}: Otimizados para Helmholtz
\item \textbf{Métodos híbridos}: Combinam diferentes abordagens
\end{itemize}
\end{block}
\end{frame}

\begin{frame}{Resultados Típicos na Literatura}
\begin{block}{Padrões Observados}
\begin{itemize}
\item \textbf{Ordem 2}: Taxa de convergência $p \approx 2.0$ para $k$ pequeno (LeVeque, 2007)
\item \textbf{Poluição}: Taxa degrada para $p < 1$ quando $kh$ grande (Babuška \& Sauter, 1997)
\item \textbf{Requisito de malha}: $N_{\lambda} \geq 10$ mínimo, $N_{\lambda} \geq 20$ recomendado (Ihlenburg \& Babuška, 1995)
\item \textbf{Erros relativos}: $10^{-6}$ a $10^{-8}$ alcançáveis para $k$ pequeno
\end{itemize}
\end{block}
\end{frame}

\begin{frame}{Nossos Resultados vs Literatura}
\begin{columns}
\begin{column}{0.5\textwidth}
\begin{block}{Comparação}
\begin{itemize}
\item Taxa experimental: $p \approx 2.0$ para $k=1$ (conforme esperado - LeVeque, 2007)
\item Taxa degrada para $p \approx 0.5-1.0$ para $k=100$ (poluição numérica - Babuška \& Sauter, 1997)
\item Erros relativos: $10^{-7}$ a $10^{-8}$ para $k=1$ (excelente)
\item Comportamento consistente com literatura
\end{itemize}
\end{block}
\end{column}

\begin{column}{0.5\textwidth}
\begin{block}{Referências Principais}
\tiny
\begin{itemize}
\item \textbf{Babuška, I., \& Sauter, S.} (1997). Is the pollution effect of the FEM avoidable for the Helmholtz equation considering high wave numbers? \textit{SIAM Review}, 39(3), 484-509.

\vspace{0.15cm}

\item \textbf{Ihlenburg, F., \& Babuška, I.} (1995). Finite element solution of the Helmholtz equation with high wave number part I: The $h$-version of the FEM. \textit{Computers \& Mathematics with Applications}, 30(9), 9-37.

\vspace{0.15cm}

\item \textbf{LeVeque, R. J.} (2007). \textit{Finite Difference Methods for Ordinary and Partial Differential Equations: Steady-State and Time-Dependent Problems}. SIAM.
\end{itemize}
\end{block}
\end{column}
\end{columns}
\end{frame}
