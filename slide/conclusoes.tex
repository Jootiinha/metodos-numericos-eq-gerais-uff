% Seção 4: Conclusões

\section{Conclusões}

\begin{frame}{Conclusões}
\begin{block}{Resultados Principais}
\begin{itemize}
\item O MDFC2 é adequado para baixos valores de $k$ ($k \leq 20$)
\item Taxa de convergência confirma ordem 2 para $k$ pequeno ($p \approx 2.0$)
\item O efeito de poluição numérica limita a eficiência para $k$ grande
\item Métodos de ordem superior reduzem esse efeito (perspectiva futura)
\item IA mostrou-se útil como suporte acadêmico
\end{itemize}
\end{block}

\vspace{0.2cm}

\begin{block}{Limitações Identificadas}
\begin{itemize}
\item \textbf{Poluição numérica}: Mesmo com refinamento, erro cresce com $k$ grande
\item \textbf{Requisito de malha}: Para $k=100$, necessário $N_{\lambda} \geq 20-30$ pontos/comprimento de onda
\item \textbf{Custo computacional}: Cresce com $N^2$ (sistemas grandes requerem métodos iterativos)
\item \textbf{Precisão limitada}: Erro relativo $> 1.0$ para $k=100$ com $N=60$
\end{itemize}
\end{block}
\end{frame}

\begin{frame}{Validação do Método}
\begin{block}{Conformidade com Teoria}
\begin{itemize}
\item \textbf{Consistência}: Erro de truncamento $O(h^2)$ confirmado numericamente
\item \textbf{Convergência}: Taxa experimental $p \approx 2.0$ para $k$ pequeno (conforme esperado)
\item \textbf{Estabilidade}: Sistema bem-condicionado para $k$ pequeno/médio
\item \textbf{Precisão}: Erros relativos da ordem de $10^{-7}$ a $10^{-8}$ para $k=1$
\end{itemize}
\end{block}

\vspace{0.2cm}

\begin{block}{Comparação com Literatura}
\begin{itemize}
\item Resultados consistentes com trabalhos sobre diferenças finitas para Helmholtz
\item Comportamento de poluição numérica observado conforme esperado
\item Requisitos de malha ($N_{\lambda} \geq 10-20$) alinhados com literatura
\item Taxa de convergência degradada para $k$ grande documentada
\end{itemize}
\end{block}
\end{frame}

\begin{frame}{Perspectivas Futuras}
\begin{block}{Melhorias Metodológicas}
\begin{itemize}
\item \textbf{Métodos de alta ordem}: 4ª ou 6ª ordem para reduzir poluição numérica
\item \textbf{Métodos híbridos}: Combinação de diferentes abordagens
\item \textbf{Malhas adaptativas}: Refinamento local onde necessário
\item \textbf{Pré-condicionadores especializados}: Otimizados para equação de Helmholtz
\end{itemize}
\end{block}

\vspace{0.2cm}

\begin{block}{Integração com IA}
\begin{itemize}
\item \textbf{IA + Métodos Numéricos}: Uso de IA para otimização de parâmetros
\item \textbf{Seleção automática de métodos}: IA sugere melhor método para cada caso
\item \textbf{Análise preditiva}: IA prevê comportamento numérico antes da execução
\end{itemize}
\end{block}

\vspace{0.2cm}

\begin{block}{Extensões}
\begin{itemize}
\item \textbf{Problemas 3D}: Generalização para domínios tridimensionais
\item \textbf{Condições de contorno avançadas}: Robin, Neumann, PML
\item \textbf{Comparação sistemática}: Elementos finitos, métodos espectrais
\end{itemize}
\end{block}
\end{frame}
