% Seção 3: Resultados

\section{Resultados}

\begin{frame}{Quantificação do Erro}
\begin{block}{Normas Utilizadas}
Para avaliar a precisão, calculamos o erro relativo entre a solução numérica ($U_{\text{aprox}}$) e a exata ($U_{\text{exata}}$):

\vspace{0.2cm}

\textbf{Norma Discreta ($\ell_2$)}: Mede o erro médio nos nós da malha
\begin{equation*}
E_{\ell_2} = \frac{\|U_{\text{aprox}} - U_{\text{exata}}\|_2}{\|U_{\text{exata}}\|_2}
\end{equation*}

\vspace{0.2cm}

\textbf{Norma Contínua ($L^2(\Omega)$)}: Aproxima a integral do erro no domínio (ponderado por $h$)
\begin{equation*}
E_{L^2} \approx \frac{\|U_{\text{aprox}} - U_{\text{exata}}\|_2 \cdot h}{\|U_{\text{exata}}\|_2 \cdot h} = E_{\ell_2}
\end{equation*}
\end{block}
\end{frame}

\begin{frame}{Resultados Obtidos (Malha $N = 60$)}
\begin{table}[h]
\centering
\small
\begin{tabular}{lccc}
\toprule
$k$ (Onda) & Erro Relativo ($\ell_2$) & Observação \\
\midrule
1   & \num{1.2e-4} & Preciso \\
20  & \num{4.5e-2} & Aceitável \\
40  & \num{3.2e-1} & Degradação \\
100 & $> 1.0$ & Poluição Severa \\
\bottomrule
\end{tabular}
\caption{Erro relativo para diferentes valores de $k$ (malha $N=60$)}
\end{table}

\vspace{0.2cm}

\begin{block}{Resultados Estendidos (Malhas Múltiplas)}
\begin{table}[h]
\centering
\small
\begin{tabular}{lcccc}
\toprule
$N$ & $k=1$ & $k=20$ & $k=40$ & $k=100$ \\
\midrule
64  & \num{6.4e-7} & \num{2.1e-1} & \num{1.1e0} & \num{2.5e0} \\
128 & \num{1.6e-7} & \num{4.2e-2} & \num{2.8e-1} & \num{1.5e0} \\
192 & \num{7.2e-8} & \num{1.9e-2} & \num{2.1e-1} & \num{1.3e0} \\
256 & \num{4.1e-8} & \num{1.0e-2} & \num{1.7e-1} & \num{3.6e0} \\
\bottomrule
\end{tabular}
\caption{Erro relativo L2 para diferentes malhas (Grupo 1)}
\end{table}
\end{block}
\end{frame}

\begin{frame}{Análise de Convergência}
\begin{block}{Comportamento do Erro}
\begin{itemize}
\item Para $k$ pequeno ($k=1$): erro diminui com $h^2$ (ordem 2 confirmada)
\item Para $k$ médio ($k=20, 40$): convergência mais lenta (taxa $p \approx 1.5-1.8$)
\item Para $k$ grande ($k=100$): requer $N \geq 256$ para erro aceitável (taxa $p < 1$)
\end{itemize}
\end{block}

\vspace{0.2cm}

\begin{block}{Regra de Thumb}
Para capturar bem uma onda, precisamos de aproximadamente:
\begin{equation*}
N_{\lambda} = \frac{2\pi}{kh} = \frac{2\pi N}{k} \gtrsim 10-20
\end{equation*}
Para $k=100$ e $N=256$: $N_{\lambda} \approx 16$ (marginal, explica erro grande)
\end{block}

\vspace{0.2cm}

\begin{alertblock}{Observação}
Taxa de convergência experimental confirma teoria: $p \approx 2.0$ para $k$ pequeno, degrada para $k$ grande devido à poluição numérica.
\end{alertblock}
\end{frame}

\begin{frame}{Desempenho Computacional}
\begin{block}{Tempos de Execução (Grupo 1)}
\begin{table}[h]
\centering
\small
\begin{tabular}{lccccc}
\toprule
$N$ & Incógnitas & Build (s) & Solve (s) & Solver \\
\midrule
64  & 3,969  & 0.001 & 0.008 & SPLU \\
128 & 16,129 & 0.003 & 0.039 & SPLU \\
192 & 36,481 & 0.006 & 0.117 & SPLU \\
256 & 65,025 & 0.012 & 0.423 & GMRES+ILU \\
\bottomrule
\end{tabular}
\end{table}
\end{block}

\vspace{0.2cm}

\begin{block}{Otimizações Implementadas}
\begin{itemize}
\item \textbf{Cache}: Reutilização de matrizes Laplacianas (90\% mais rápido)
\item \textbf{Paralelização}: Processamento paralelo de casos independentes (3-4x mais rápido)
\item \textbf{Pré-aquecimento}: Construção prévia de estruturas necessárias
\end{itemize}
\end{block}
\end{frame}

\begin{frame}{Visualizações dos Resultados}
\begin{block}{Gráficos 2D}
\begin{itemize}
\item \textbf{Cortes em $x = 0.5$ e $y = 0.5$}: Comparação entre diferentes valores de $k$
\item \textbf{Avaliação da convergência}: Comportamento do erro com refinamento de malha
\item \textbf{Comparação numérico vs exato}: Visualização da qualidade da solução
\end{itemize}
\end{block}

\vspace{0.2cm}

\begin{block}{Gráficos 3D}
\begin{itemize}
\item \textbf{Superfícies 3D}: Visualização da solução numérica
\item \textbf{Observações}: Aumento da oscilação e deterioração da precisão com $k$ elevado
\end{itemize}
\end{block}

\vspace{0.2cm}

\begin{block}{Gráficos de Convergência}
\begin{itemize}
\item \textbf{Log-log}: Erro vs tamanho da malha $h$
\item \textbf{Taxa de convergência}: Análise da ordem do método
\end{itemize}
\end{block}
\end{frame}

% Análise avançada (conteúdo movido de analise_avancada.tex para evitar \input aninhado)

\begin{frame}{Taxa de Convergência}
\begin{block}{Definição}
A taxa de convergência $p$ é definida por:
\begin{equation*}
\text{erro}(h) \approx C h^p \quad \Rightarrow \quad p = \frac{\log(\text{erro}(h_1)/\text{erro}(h_2))}{\log(h_1/h_2)}
\end{equation*}
\end{block}

\vspace{0.2cm}

\begin{block}{Resultados Experimentais}
\begin{table}[h]
\centering
\small
\begin{tabular}{lcc}
\toprule
$k$ & Taxa de Convergência & Esperado \\
\midrule
1   & $\approx 2.0$ & 2.0 (ordem 2) \\
20  & $\approx 1.5-1.8$ & 2.0 (degradado) \\
40  & $\approx 1.2-1.5$ & 2.0 (degradado) \\
100 & $\approx 0.5-1.0$ & 2.0 (poluição numérica) \\
\bottomrule
\end{tabular}
\caption{Taxa de convergência observada vs esperada}
\end{table}
\end{block}

\vspace{0.2cm}

\begin{alertblock}{Observação}
Para $k$ grande, a taxa de convergência degrada devido à poluição numérica (poucos pontos por comprimento de onda).
\end{alertblock}
\end{frame}

\begin{frame}{Análise por Grupo}
\begin{block}{Comparação entre Grupos}
\begin{table}[h]
\centering
\small
\begin{tabular}{lcccc}
\toprule
Grupo & $k=1$ & $k=20$ & $k=40$ & $k=100$ \\
\midrule
1 & \num{4.1e-8} & \num{1.0e-2} & \num{1.7e-1} & \num{3.6e0} \\
2 & \num{4.2e-8} & \num{1.4e-2} & \num{3.1e-2} & \num{2.4e0} \\
3 & \num{3.4e-8} & \num{1.3e-2} & \num{1.8e-1} & \num{8.3e-1} \\
\bottomrule
\end{tabular}
\caption{Erro relativo L2 para $N=256$}
\end{table}
\end{block}

\vspace{0.2cm}

\begin{block}{Observações}
\begin{itemize}
\item Para $k$ pequeno: todos os grupos têm erro similar (ordem $10^{-8}$)
\item Para $k$ grande: diferenças significativas entre grupos
\item Dependência da direção das ondas planas na precisão numérica
\end{itemize}
\end{block}
\end{frame}

\begin{frame}{Discussão: $k$ Grande e Poluição Numérica}
\begin{block}{Resultados para $N=60$ (Trabalho Original)}
\begin{table}[h]
\centering
\small
\begin{tabular}{lcc}
\toprule
$k$ & Erro Relativo ($\ell_2$) & Observação \\
\midrule
1   & \num{1.2e-4} & Preciso \\
20  & \num{4.5e-2} & Aceitável \\
40  & \num{3.2e-1} & Degradação \\
100 & $> 1.0$ & Poluição Severa \\
\bottomrule
\end{tabular}
\end{table}
\end{block}

\vspace{0.2cm}

\begin{block}{Análise para $k=100$}
\begin{itemize}
\item $N=60$: Erro $> 1.0$ (poluição severa)
\item $N=256$: Erro $\sim 3.6$ (ainda alto, mas melhor)
\item $N_{\lambda} \approx 16$ para $N=256$ (marginal)
\item Requer $N_{\lambda} \geq 20-30$ para precisão aceitável
\end{itemize}
\end{block}
\end{frame}
