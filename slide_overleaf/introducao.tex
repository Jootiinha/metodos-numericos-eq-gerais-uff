% Seção 1: Introdução

\section{Introdução}

\begin{frame}{Equação de Helmholtz}
\begin{block}{Equação de Helmholtz 2D}
A equação de Helmholtz é uma equação diferencial parcial elíptica que surge naturalmente na modelagem de fenômenos ondulatórios em regime harmônico:
\begin{equation*}
\nabla^2 u + k^2 u = 0 \quad \text{em } \Omega = [0,1]^2
\end{equation*}
\end{block}

\vspace{0.3cm}

\begin{block}{Interpretação dos Termos}
\begin{itemize}
\item $\nabla^2 u$: Operador Laplaciano — mede a curvatura espacial
\item $k^2 u$: Termo reativo — impõe caráter oscilatório
\item $u(x,y)$: Campo escalar desconhecido
\end{itemize}
\end{block}
\end{frame}

\begin{frame}{Observação Importante}
\begin{alertblock}{Dependência do Número de Onda}
Quanto maior o valor de $k$, maior a frequência espacial da solução, exigindo malhas mais refinadas.
\end{alertblock}

\vspace{0.3cm}

\begin{block}{Desafio Numérico}
\begin{itemize}
\item Para $k$ grande: requer muitas incógnitas
\item Sistemas lineares grandes
\item Necessidade de métodos eficientes
\end{itemize}
\end{block}
\end{frame}

\begin{frame}{Derivação - Equação da Onda}
\begin{block}{Equação da Onda}
\begin{equation*}
\frac{\partial^2 u}{\partial t^2} = c^2 \nabla^2 u
\end{equation*}
\end{block}

\vspace{0.3cm}

\begin{block}{Solução Harmônica}
Assumindo solução periódica no tempo:
\begin{equation*}
u(x, y, t) = \text{Re}\left(U(x, y) e^{-i\omega t}\right)
\end{equation*}
\end{block}
\end{frame}

\begin{frame}{Derivação - Equação de Helmholtz}
\begin{block}{Substituição}
Substituindo na equação da onda:
\begin{equation*}
\nabla^2 U + \left(\frac{\omega}{c}\right)^2 U = 0
\end{equation*}
\end{block}

\vspace{0.3cm}

\begin{block}{Número de Onda}
Definindo $k = \omega/c$, obtém-se a equação de Helmholtz:
\begin{equation*}
\nabla^2 U + k^2 U = 0
\end{equation*}
\end{block}
\end{frame}

\begin{frame}{Solução Exata}
\begin{block}{Superposição de Ondas Planas}
Para validação e cálculo de erro, utilizamos solução exata imposta como condição de Dirichlet na fronteira:
\begin{equation*}
u_{\text{exata}}(x, y) = \sum_{i=1}^{3} \cos\left(k (x \cos \theta_i + y \sin \theta_i)\right)
\end{equation*}
com ângulos: $\Theta = \{0, \pi/8, \pi/4\}$
\end{block}

\vspace{0.2cm}

\begin{block}{Aplicações Práticas}
\begin{itemize}
\item Acústica arquitetônica
\item Propagação eletromagnética
\item Vibrações mecânicas
\item Difração e espalhamento de ondas
\end{itemize}
\end{block}
\end{frame}

\begin{frame}{Motivação}
\begin{columns}
\begin{column}{0.5\textwidth}
\begin{block}{Aplicações}
\begin{itemize}
\item Propagação de ondas
\item Acústica
\item Eletromagnetismo
\item Óptica
\end{itemize}
\end{block}
\end{column}

\begin{column}{0.5\textwidth}
\begin{block}{Desafios Numéricos}
\begin{itemize}
\item Alta frequência ($k$ grande)
\item Requer malhas refinadas
\item Sistemas lineares grandes
\item Matrizes esparsas
\end{itemize}
\end{block}
\end{column}
\end{columns}
\end{frame}

\begin{frame}{Objetivo Geral}
\begin{block}{Objetivo Principal}
Aplicar o Método das Diferenças Finitas Centradas de Segunda Ordem (MDFC2) à equação de Helmholtz bidimensional e analisar seus efeitos numéricos.
\end{block}
\end{frame}

\begin{frame}{Objetivos Específicos}
\begin{block}{Metas do Trabalho}
\begin{itemize}
\item Estudar a formulação matemática da equação de Helmholtz
\item Implementar a discretização por MDFC2
\item Investigar a influência do número de onda $k$
\item Analisar o fenômeno de poluição numérica
\end{itemize}
\end{block}
\end{frame}

\begin{frame}{Objetivos Específicos (continuação)}
\begin{block}{Metas Adicionais}
\begin{itemize}
\item Avaliar o uso de Inteligência Artificial como ferramenta auxiliar
\item Comparar métodos considerando precisão e custo computacional
\end{itemize}
\end{block}
\end{frame}
