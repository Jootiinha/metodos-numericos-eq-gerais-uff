% Seção: Fundamentação Teórica

\section{Fundamentação Teórica}

\begin{frame}{Ideia Geral do Método das Diferenças Finitas}
\begin{block}{Conceito}
O Método das Diferenças Finitas consiste em aproximar derivadas por combinações lineares de valores da função em pontos discretos de uma malha uniforme.
\end{block}

\vspace{0.3cm}

\begin{block}{Vantagens}
\begin{itemize}
\item Simplicidade de implementação
\item Estrutura de matriz esparsa
\item Fácil paralelização
\item Boa para domínios retangulares
\end{itemize}
\end{block}
\end{frame}

\begin{frame}{Malha Uniforme e Stencil}
\begin{block}{Malha Uniforme}
\begin{itemize}
\item Domínio $\Omega = [0,1]^2$ discretizado em malha uniforme
\item $N$ subintervalos em cada direção
\item Tamanho do passo: $h = 1/N$
\item Pontos da malha: $(x_i, y_j) = (ih, jh)$
\end{itemize}
\end{block}

\vspace{0.3cm}

\begin{block}{Stencil de 5 Pontos}
Para discretizar o Laplaciano 2D, utilizamos um estêncil de 5 pontos (centro + 4 vizinhos).
\end{block}
\end{frame}

\begin{frame}{Diferença Finita Centrada - Primeira Derivada}
\begin{block}{Aproximação}
A derivada primeira pode ser aproximada por:
\begin{equation*}
\frac{\partial u}{\partial x} \approx \frac{u(x+h) - u(x-h)}{2h}
\end{equation*}
\end{block}

\vspace{0.3cm}

\begin{block}{Propriedades}
Essa aproximação:
\begin{itemize}
\item Utiliza pontos simétricos (centrada)
\item Possui erro de truncamento $O(h^2)$
\item É mais precisa que diferenças progressivas ou regressivas
\end{itemize}
\end{block}
\end{frame}

\begin{frame}{Diferença Finita Centrada - Segunda Derivada}
\begin{block}{Aproximação}
A segunda derivada é aproximada por:
\begin{equation*}
\frac{\partial^2 u}{\partial x^2} \approx \frac{u(x+h) - 2u(x) + u(x-h)}{h^2}
\end{equation*}
\end{block}

\vspace{0.3cm}

\begin{block}{Importância}
Essa forma é fundamental para a discretização do operador Laplaciano.
\end{block}
\end{frame}

\begin{frame}{Discretização do Laplaciano 2D}
\begin{block}{Laplaciano Contínuo}
\begin{equation*}
\Delta u = \frac{\partial^2 u}{\partial x^2} + \frac{\partial^2 u}{\partial y^2}
\end{equation*}
\end{block}

\vspace{0.3cm}

\begin{block}{Discretização com Stencil de 5 Pontos}
Em uma malha uniforme $(x_i, y_j) = (ih, jh)$:
\begin{equation*}
\Delta u(x_i, y_j) \approx \frac{u_{i+1,j} - 2u_{i,j} + u_{i-1,j}}{h^2} + \frac{u_{i,j+1} - 2u_{i,j} + u_{i,j-1}}{h^2}
\end{equation*}
\end{block}
\end{frame}

\begin{frame}{Equação de Helmholtz Discretizada}
\begin{block}{Aplicando MDFC2}
Substituindo na equação de Helmholtz:
\begin{equation*}
\frac{u_{i+1,j} - 2u_{i,j} + u_{i-1,j}}{h^2} + \frac{u_{i,j+1} - 2u_{i,j} + u_{i,j-1}}{h^2} + k^2 u_{i,j} = 0
\end{equation*}
\end{block}

\vspace{0.3cm}

\begin{block}{Sistema Linear}
Resulta em um sistema linear esparso:
\begin{equation*}
A \mathbf{u} = \mathbf{b}
\end{equation*}
\end{block}
\end{frame}

\begin{frame}{Consistência}
\begin{block}{Definição}
O método é \textbf{consistente} de ordem 2:
\begin{equation*}
\tau_{i,j} = O(h^2) \quad \text{(erro de truncamento local)}
\end{equation*}
\end{block}

\vspace{0.3cm}

\begin{block}{Consequência}
Aproximação converge para a solução exata quando $h \to 0$.
\end{block}
\end{frame}

\begin{frame}{Estabilidade}
\begin{block}{Sistema Linear}
Para a equação de Helmholtz:
\begin{equation*}
A\mathbf{u} = \mathbf{b}
\end{equation*}
\end{block}

\vspace{0.2cm}

\begin{block}{Propriedades da Matriz}
\begin{itemize}
\item Matriz $A = L + k^2 I$ é definida positiva para $k$ real
\item Condicionamento piora com $k$ grande (requer malhas mais finas)
\end{itemize}
\end{block}
\end{frame}

\begin{frame}{Teorema de Lax}
\begin{alertblock}{Teorema Fundamental}
Consistência + Estabilidade $\Rightarrow$ Convergência
\end{alertblock}

\vspace{0.3cm}

\begin{block}{Aplicação}
\begin{itemize}
\item Método MDFC2 é consistente de ordem 2
\item Sistema é estável para $k$ real
\item Portanto, o método converge com ordem 2
\end{itemize}
\end{block}
\end{frame}

\begin{frame}{Dispersão Numérica}
\begin{block}{Problema}
O método de diferenças finitas introduz \textbf{dispersão numérica}: a velocidade de fase numérica difere da exata.
\end{block}

\vspace{0.3cm}

\begin{block}{Consequência}
\begin{itemize}
\item Erro de fase acumula
\item Solução numérica "atrasa" espacialmente
\item Taxa de convergência degrada ($p < 2$)
\end{itemize}
\end{block}
\end{frame}

\begin{frame}{Número de Pontos por Comprimento de Onda}
\begin{block}{Definição}
Para capturar uma onda com número de onda $k$:
\begin{equation*}
N_{\lambda} = \frac{2\pi}{kh} = \frac{2\pi N}{k}
\end{equation*}
\end{block}

\vspace{0.3cm}

\begin{block}{Recomendação}
$N_{\lambda} \geq 10-20$ pontos por comprimento de onda.
\end{block}

\vspace{0.2cm}

\begin{alertblock}{Poluição Numérica}
Para $k$ grande e $h$ fixo, temos poucos pontos por comprimento de onda, causando poluição numérica.
\end{alertblock}
\end{frame}

\begin{frame}{Erro de Truncamento Local}
\begin{block}{Expansão de Taylor}
Para diferenças centradas de 2ª ordem:
\begin{equation*}
\frac{\partial^2 u}{\partial x^2}\Big|_{x_i} = \frac{u_{i+1} - 2u_i + u_{i-1}}{h^2} - \frac{h^2}{12}\frac{\partial^4 u}{\partial x^4}\Big|_{\xi} + O(h^4)
\end{equation*}
\end{block}

\vspace{0.3cm}

\begin{block}{Erro de Truncamento}
$\tau = O(h^2)$ (ordem 2)
\end{block}
\end{frame}

\begin{frame}{Erro Global e Convergência}
\begin{block}{Estimativa de Erro}
Pelo teorema de Lax (consistência + estabilidade):
\begin{equation*}
\|u - u_h\|_{L^2} \leq C h^2 \|u\|_{H^4}
\end{equation*}
\end{block}

\vspace{0.3cm}

\begin{block}{Ordem de Convergência}
$p = 2$ (assintoticamente)
\end{block}

\vspace{0.2cm}

\begin{alertblock}{Nota}
Para $k$ grande, a constante $C$ pode crescer, e o regime assintótico requer $h$ muito pequeno.
\end{alertblock}
\end{frame}
